\documentclass{bioinfo}
\copyrightyear{2015} \pubyear{2015}

\access{Advance Access Publication Date: Day Month Year}
\appnotes{Manuscript Category}

\begin{document}
\firstpage{1}

\subtitle{Subject Section}

\title[short Title]{SSR-viz - a toolbox to detect and visualize protein
subfamily specific residues}
\author[Sample \textit{et~al}.]{Paul Zierep\,$^{\text{\sfb 1,}*}$, Stefan G\"unther\,$^{\text{\sfb 2}}$}
\address{$^{\text{\sf 1}}$Pharmaceutical Bioinformatics, Institute of Pharmaceutical Science, Albert-Ludwigs-University, Hermann-Herder-Strasse 9, Freiburg 79104, Germany. \\
%$^{\text{\sf 2}}$Department, Institution, City, Post Code, Country.
}

\corresp{$^\ast$To whom correspondence should be addressed.}

\history{Received on XXXXX; revised on XXXXX; accepted on XXXXX}

\editor{Associate Editor: XXXXXXX}

\abstract{
\textbf{Motivation:}
Protein subfamilies share a common structure but differ in functionalities such
as substrate specificity. These differences can often be assigned to a limited set 
of subfamily specific residues (SSRs). Often, only experimental evaluation, expert knowledge
and detailed protein structure investigation, can lead to the correct assignment of these SSRs.
% Additionally, the detected SSRs which best distinguish two subfamily classes, 
% can be very different from the SSRs needed to distinguish another class.
% In cases such as nonribosomal peptide syntheses (NRPS), where dozens of subfamily classes can be defined,
% computational guidance becomes desirable. 
% Lastly, a general applicable algorithm to detect SSRs is difficult to define, as evolutionary distances 
% and general protein diversity require fine adjustment of any algorithm to distinguish true function 
% defining SSRs from merely random mutation. \\
Here, we introduce the toolbox SSR-viz, which allows 
to detect and visualize these SSRs
based on a multiple sequence alignment (MSA) and a file containing the subfamily class labels.  
The applied algorithm can be adjusted to
a given problem and allows users to perform a thorough analysis of their sequences. \\
% The toolbox can produce three different types of outputs which allow to further analyze the results: 
% First, an overview plot which is highly customizable and allows to visualize the SSRs of the entire 
% alignment, including general regions of interest. Secondly, a Jalview Annotation File, which allows
% to investigate the results together with the alignment in a sophisticated alignment editor. Finally,
% a summary CSV-file, which shows the conserved amino acid types in each class for highest scoring SSRs.
% This file allows the inclusion of protein structure indices, which makes it easy to investigate the found
% SSRs in an structural context. 
\textbf{Results:} Substrate specificity of non-ribosomal peptide synthases were analyzed.
The detected residues are located in the binding side and are in accordance with previous
performed studies.
\\
\textbf{Availability:} The toolbox is written as a GUI in Python3 and is available via
PyPi and the GitHub repository (todo). Standalone executables are also available for 
Windows and Ubuntu.
\\
\textbf{Contact:} \href{stefan.guenther@pharmazie.uni-freiburg.de }{stefan.guenther@pharmazie.uni-freiburg.de }\\
\textbf{Supplementary information:} Supplementary data are available at \textit{Bioinformatics}
online.} 

\maketitle

\section{Introduction}

Protein subfamilies differ from each other in very specific functions, such as substrate specificity,
protein-protein interaction or reaction type. These functionalities can often be assigned 
to a limited set of residues (referred to as subfamily specific residues (SSRs) in this article).
The identification of those SSRs is crucial to the understanding of protein sub-family diversity.
SSRs with high statistical significance can form the basis for various experiments 
such as side directed mutagenesis and rational design of proteins. \\
%to alter 
%proteins in their function or rational design experiments to design protein with totally new functionality.
%
Various approaches have been proposed to detect SSRs based on analysis of multiple sequence 
alignments (MSAs) of two or more protein subfamilies (todo cite). All these approaches share the 
common objective: To identify positions in an alignment which are conserved in one 
protein subfamily but differ between subfamilies. Some algorithms also include
physico-chemical properties of the amino acids in order to distinguish
true SSRs from residues which are due to evolutionary mutations. \\
%
In order to allow researchers a thorough study of their proteins we implemented an algorithm 
which follows this common objective but is also flexible and can be adjusted to the specific 
protein family observed.
Additionally, SSR-viz provides assistance for various tasks commonly encountered on the search for SSRs.
The main features are: The easy mapping of the alignment with a CSV-file which holds the information of 
the subfamily class labels. The addition of a window function, which allows to observe
single amino acid residues as well as general areas of interest in an alignment.
The residues can be mapped to a protein structure, which facilitates the
further investigation in a protein visualization tool such as PyMol (cite).
The creation of a Jalview (cite) annotation file which allows to integrate the results 
in the Jalview Alignment Editor together with the original alignment.
%
\section{Algorithm}
%
% \begin{equation}
% PPM =
% \left[ \begin{array}{cccc}
%   \alpha_{A1} & \alpha_{A2}  & ... & \alpha_{Am} \\
%   \alpha_{R1} & \alpha_{R2}  & ... & \alpha_{Rm} \\
%   ... & ... & ... & ...\\
%   \alpha_{n1} & \alpha_{n2}  & ... & \alpha_{nm} \\
% \end{array} \right]
% \end{equation}

% \begin{equation}
% EP(m) =
%   \begin{bmatrix}
%     \alpha_{A1} * \beta_{A1} & \alpha_{A1} * \beta_{R1} & ... & \alpha_{A1} * \beta_{n1} \\
%     \alpha_{R1} * \beta_{A1} & \alpha_{R1} * \beta_{R1} & ... & \alpha_{R1} * \beta_{n1} \\
%   ... & ... & ... & ...\\
%     \alpha_{n1} * \beta_{A1} & \alpha_{n1} * \beta_{R1} & ... & \alpha_{n1} * \beta_{n1} \\
%   \end{bmatrix}
% \end{equation}
%
Initially, each subfamily class in the alignment is converted into a position 
probability matrix (PPM), which represents the probability to find a specific 
amino acid at a given position. This allows to compare the subfamily classes 
independent of the number of their representing sequences. 
Each position in the alignment is assigned to a final score based on the combination 
of three independent scoring functions.
%
The conservedness of each position is defined by the normalized Shannon entropy (see Equation~(\ref{eq:01})).
%
\begin{equation}
S(m) = 1 - \frac{-\sum(\alpha_m*\log_2(\alpha_m))}{max(S(\alpha))} \label{eq:01} %\vspace*{-10pt}
\end{equation}  
%
The difference between two classes is computed by summation of an exchange matrix. The matrix
represents the probability of each amino acid in one class to be exchanged with another amino acid in 
the other classes. The more amino acids are exchanged the higher the score.
This matrix judges each amino acid exchange equally.
%
In order to differentiate the exchange of divers amino acids,
an additional scoring function is implemented. Again the exchange matrix is computed, but
this exchange matrix can be weighed with a substitution matrix such as
Pam or Blossum. A purely physico-chemical substitution matrix is also
available based on the research of (cite).
%
An detailed explanation of the algorithm including an 
hypothetical example is given in the Supplementary data.
%\begin{methods}
\section{Implementation}
%
SSR-viz is based on a MSA as input, additionally a CSV-file is needed which holds the 
subfamily specific label of each sequence. The CSV-File can be created using the
\textbf{CSV\_builder} tool. This tool allows also the label extraction from
the sequence name using regular expressions.
The CSV-file can have multiple labels, in case different types of functionality 
are investigated. The \textbf{CSV\_builder} also excludes redundant sequences by default.\\
%
The detection of SSRs is handled by the \textbf{SSR\_plot} tool. 
SSRs can be computed in three different kind of schemes: One-vs-one,
which returns the SSRs to distinguish each class from another.
One-vs-all, which returns SSRs to distinguish one class from all the others classes,
and all-vs-all, which returns the SSRs to distinguish all the classes from each other.\\
%
The detection threshold for SSRs can be assigned in two ways. Either the top N 
SSRs can be returned independent of their significance or the SSRs can be returned based
on their Z-score, which only returns positions that have a significant higher score then 
the other positions in the alignment.\\
%
SSR-viz supports three different kind of outputs:
A mathplotlib-style plot in PDF Format, which is highly customizable and consists
of a heatmap, which visualizes all the scores and a plot which shows the most 
significant SSRs. A window function can also be added to the plot, which allows 
to investigate the alignment for areas of general importance. \\
%
It is often desired to visualize the SSRs together with the MSA, therefore,
an Jalview annotation file can also be created, this File can be loaded into 
the Jalview Alignment Editor.\\
%
Finally a summary CSV-file can be created, this file shows the SSRs
together with the conserved positions in each subfamily class. 
In order to investigate the SSRs in a structural context,
we also implemented a tool \textbf{Add\_pdb}, which allows to 
map protein structure indices from a PDB File to the indices in the MSA.
\textbf{Add\_pdb} supports any PDB file, downloaded from PDB or custom exported from 
PyMol, as long as it can be aligned to the MSA.
A flow chart of the Implementation is shown in Figure~1\vphantom{\ref{fig:01}}.

\begin{figure}[!tpb]%figure1
\fboxsep=0pt\colorbox{gray}{\begin{minipage}[t]{235pt} \vbox to 100pt{\vfill\hbox to
235pt{\hfill\fontsize{24pt}{24pt}\selectfont FPO\hfill}\vfill}
\end{minipage}}
%\centerline{\includegraphics{}}
\caption{General flow chart of the SSR-viz toolbox. a) The input consists 
of a MSA together with a CSV-file holding the corresponding subfamily class labels. 
Multiple protein structures can also be included in order to assign the 
structure indices. b) The output can be created as a mathplotlib PDF, a Jalview annotation file
or a summary CSV-File. This file is used to assign the indices of the structure.
c) Example assignment of SSRs for the NRPS adenylation domain with specificity for Valine and Glycine.   
}\label{fig:01}
\end{figure}

%\end{methods}

\section{Use case}

The toolbox was applied to detect the SSRs for the adenylation domain of   
non-ribosomal peptide synthases (NRPS). 
Two subfamilies were chosen, with respective specificities 
for valine and glycine. 
For both subfamilies crystal structures are available (PDB: 3VNS, 4ZXI),
so that the detected SSRs can be interpreted in a structural context. 
The 10 most important SSRs to distinguish the two subfamilies from each other 
were calculated (using the default parameters of ssrviz). 
Additionally the most important SSRs
to distinguish all the major subfamilies (29 subfamiles, see supplementary data)
were also calculated (see Table~\ref{Tab:01}). \\
%
\begin{table}[!t]
\processtable{10 most important SSRs to distinguish subfamilies with glycine and valine specificity and 
all the subfamily classes \label{Tab:01}} 
{\begin{tabular}{@{}lllllll@{}}\toprule G vs V & & & & All vs all &  \\
Position  & Score & G cons. & V cons. & Position & Score \\\midrule
486 & 0.74& I & A & 486 & 0.59\\
749 & 0.70& W & F & 749 & 0.57\\
478 & 0.68& T & S & 739 & 0.56\\
601 & 0.66& W & F & 591 & 0.55\\
748 & 0.65& I & T & 490 & 0.54\\
591 & 0.64& Q & W & 478 & 0.53\\
592 & 0.62& A & L & 750 & 0.53\\
404 & 0.58& F & L & 601 & 0.52\\
403 & 0.58& N & R & 495 & 0.52\\
479 & 0.57& T & N & 476 & 0.51\\
\botrule
\end{tabular}}{Note: G cons. stands for the conserved residue for the subfamily with specificity for Glycine, The Positions are
based on the alignment supplied in the supplementary data.}
\end{table}
%
All detected SSRs are in close proximity to the ligand (< 16 Angstrom). 
From the SSRs computed for the all-vs-all scheme, 6 are identical to those 
proposed by (cite Stachehauser) (749, 601, 591, 495, 490, 478).
%
Interestingly, the most significant position found by our analysis (486) is not 
included in the Stachehauser code. Nevertheless, its key role can clearly be 
derived from its position in the binding pocket (see supplementary data and Figure~1\vphantom{\ref{fig:01}}).
%
The two main positions are identical for both SSR schemes. These positions are 
also closest to the ligand in the crystal structure (see supplementary data). 
The subsequent positions differ between the schemes, highlighting the 
importance of a detailed subfamily specific investigation. 
%
\section{Conclusion}
%
SSR-viz allows for a thorough analysis of protein subfamilies for SSRs based on different schemes and using 
various different substitution matrices. %Therefore, it can be adjusted to a specific protein family. 
The diversity of the output options allows to further analyze the SSRs together with the alignment or in a structural context. 
We hope, that SSR-viz will improve the understanding of protein subfamily diversity by allowing the 
simple analysis of the responsible SSRs.
%
\section*{Funding}

This work has been supported by the... Text Text  Text Text.\vspace*{-12pt}

%\bibliographystyle{natbib}
%\bibliographystyle{achemnat}
%\bibliographystyle{plainnat}
%\bibliographystyle{abbrv}
%\bibliographystyle{bioinformatics}
%
%\bibliographystyle{plain}
%
%\bibliography{Document}


\begin{thebibliography}{}
% 
% \bibitem[Bofelli {\it et~al}., 2000]{Boffelli03}
% Bofelli,F., Name2, Name3 (2003) Article title, {\it Journal Name}, {\bf 199}, 133-154.
% 
% \bibitem[Bag {\it et~al}., 2001]{Bag01}
% Bag,M., Name2, Name3 (2001) Article title, {\it Journal Name}, {\bf 99}, 33-54.
% 
% \bibitem[Yoo \textit{et~al}., 2003]{Yoo03}
% Yoo,M.S. \textit{et~al}. (2003) Oxidative stress regulated genes
% in nigral dopaminergic neurnol cell: correlation with the known
% pathology in Parkinson's disease. \textit{Brain Res. Mol. Brain
% Res.}, \textbf{110}(Suppl. 1), 76--84.
% 
% \bibitem[Lehmann, 1986]{Leh86}
% Lehmann,E.L. (1986) Chapter title. \textit{Book Title}. Vol.~1, 2nd edn. Springer-Verlag, New York.
% 
% \bibitem[Crenshaw and Jones, 2003]{Cre03}
% Crenshaw, B.,III, and Jones, W.B.,Jr (2003) The future of clinical
% cancer management: one tumor, one chip. \textit{Bioinformatics},
% doi:10.1093/bioinformatics/btn000.
% 
% \bibitem[Auhtor \textit{et~al}. (2000)]{Aut00}
% Auhtor,A.B. \textit{et~al}. (2000) Chapter title. In Smith, A.C.
% (ed.), \textit{Book Title}, 2nd edn. Publisher, Location, Vol. 1, pp.
% ???--???.
% 
% \bibitem[Bardet, 1920]{Bar20}
% Bardet, G. (1920) Sur un syndrome d'obesite infantile avec
% polydactylie et retinite pigmentaire (contribution a l'etude des
% formes cliniques de l'obesite hypophysaire). PhD Thesis, name of
% institution, Paris, France.

\end{thebibliography}
\end{document}
