\documentclass{bioinfo}
\copyrightyear{2015} \pubyear{2015}

\access{Advance Access Publication Date: Day Month Year}
\appnotes{Manuscript Category}

\begin{document}
\firstpage{1}

\subtitle{Subject Section}

\title[short Title]{SSR-viz - a toolbox to detect and visualize protein
subfamily specific residues}
\author[Sample \textit{et~al}.]{Paul Zierep\,$^{\text{\sfb 1,}*}$, Stefan Günther\,$^{\text{\sfb 2}}$}
\address{$^{\text{\sf 1}}$Department, Institution, City, Post Code, Country and \\
$^{\text{\sf 2}}$Department, Institution, City, Post Code,
Country.}

\corresp{$^\ast$To whom correspondence should be addressed.}

\history{Received on XXXXX; revised on XXXXX; accepted on XXXXX}

\editor{Associate Editor: XXXXXXX}

\abstract{
\textbf{Motivation:}
Protein sub-families share a similar structure but differ in functionality such
as substrate specificity. These differences can often be assigned to a limited set 
of residues, which are difficult to detect from sequence data alone. 
Here, we introduce a toolbox which allows to detect and visualize these residues 
based on a multiple sequence alignment. (todo Maybe Output ??)
The applied algorithm can be adjusted to
a given problem and allows users to perform a thorough analysis of their sequences. 
\\
\textbf{Results:} Substrate specificty of non-ribosomal peptide synthases were analysed.
The detected residues are located in the binding side and are in accordance with previous
performed studies.
\\
\textbf{Availability:} The toolbox is written as a GUI in Python3 and is available via
PyPi and the GitHub repository (todo). Standalone executalbes are also available for 
Windows and Ubuntu.
\\
\textbf{Contact:} \href{name@bio.com}{name@bio.com}\\
\textbf{Supplementary information:} Supplementary data are available at \textit{Bioinformatics}
online.} 

\maketitle

\section{Introduction}

Protein sub-families differ from each other in very specific functions, such as substrate specificity,
protein-protein interaction or reaction type. These functionalities can often be assigned 
to a limited set of residues (referred to as subfamily specific residues (SSRs) in this article).
The identification of those SSRs is crucial to the understanding of protein sub-family diversity.
SSRs with high statistical significance
can form the basis for various experiments such as side directed mutagenesis and rational design. 
%to alter 
%proteins in their function or rational design experiments to design protein with totally new functionality.

Various approaches have been proposed to detect SSRs based on analysis of multiple sequence 
alignments (MSAs) of two or more protein subfamilies (todo cite). All these approaches share the 
common objective: To identify positions in an alignment which are conserved in one 
protein subfamily but differ between subfamilies. Some algorithms also include
physico-chemical properties of the amino acids in order to distinguish
true SSRs from residues which are due to evolutionary mutations. 

In order to allow researchers a thorough study of their proteins we implemented an algorithm 
which follows this common objective but is also flexible and can be adjusted to the specific 
protein family observed.

Additionally, SSR-viz provides assistance for various tasks commonly encountered on the search for SSRs.
The main features are: The easy mapping of the alignment with a CSV-file which holds the information of 
the subfamily class labels. The addition of a window function, which allows to observe
single amino acid residues as well as general areas of interest in the alignment.
The residues can be mapped to a protein structure, which facilitates the
further investigation in a protein visualization tool such as PyMol (cite).
The creation of a Jalview (cite) annotation file which allows to integrate the results 
in the Jalview Alignment Editor together with the original alignment.

\section{Algorithm}

\begin{equation}
a
\end{equation}

\begin{equation}
a
\end{equation}

\begin{equation}
a
\end{equation}

Initially, each subfamily class in the alignment is converted into a position 
probability matrix (PPM), which represents the probability to find a specific 
amino acid at a given position (see eq. ...). This allows to compare the subfamily classes 
independent of the number of their sequences. 

Each position in the alignment are assigned to a final score based on the combination 
of three independent scoring functions.

The conservedness of each position is defined by the normalized shannon entropy (see eq. ..).
The difference between two classes is computed by summation of a exchange matrix. The matrix
represents the probability of each amino acid in one class to be exchanged with another amino acid in 
the other classes. The more amino acids are exchanged the higher the score (see eq. ...).
This matrix judges each amino acid exchange equally.

In order to differentiate the exchange of divers amino acids,
an additionally scoring function is implemented similar to eq. 2, but 
this exchange matrix can be weighed with a substitution matrix such as
Pam or Blossum. A purely physico-chemical substitution matrix is also
available based on the research (?) of ....


\begin{methods}
\section{Implementation}

SSR-viz is based on a MSA as Input, additionally a CSV-File is needed which holds the 
subfamily specific label of each sequence. The CSV-File can be created using the
\textbf{CSV\_builder} tool. This tool allows also the label extraction from
the sequence name using regular expressions.
The CSV-File can have multiple labels, in case different types of functionality 
are investigated. The \textbf{CSV\_builder} also excludes redundant sequences by default.

The detection of SSRs is handled by the \textbf{SSR\_plot} tool. 
SSRs can be computed in three different kind of schemes: One-vs-one,
which returns the SSRs to distinguish each class from another.
One-vs-All, which returns SSRs to distinguish one class from all the others classes,
and All-vs-All, which returns the SSRs to distinguish all the classes from each other.

The detection threshold for SSRs can be assigned in two ways. Either the top N 
SSRs can be returned independent of their significance or the SSRs can be returned based
on their Z-score, which only returns positions that have a significant higher score then 
the other positions in the alignment.
 
SSR-viz supports three different kind of outputs:
A mathplotlib-style plot in PDF Format, which is highly customizable and consists
of a heatmap, which visualizes all the scores and a plot which shows the most 
significant SSRs. A window function can also be added to the plot, which allows 
to investigate the alignment for areas of general importance. 

It is often desired to visualize the SSRs together with the MSA, therefore,
an Jalview Annotation File can also be created, this File can be loaded into 
the Jalview Alignment Editor.

Finally a summary CSV-file can be created, this file shows the SSRs
together with the conserved positions in each subfamily class. 

It is often desired to investigate the SSRs in a structural context. 
Therefore, we also implemented a tool \textbf{Add\_pdb}, which allows to 
map protein structure indices from a PDB File to the indices in the MSA.
\textbf{Add\_pdb} supports any PDB file, downloaded from PDB or custom exported from 
PyMol, as long as they can be aligned to the MSA.

\end{methods}


\section{Use case}


\section*{Acknowledgements}

Text Text Text Text Text Text  Text Text.  \citealp{Boffelli03} might want to know about  text
text text text\vspace*{-12pt}

\section*{Funding}

This work has been supported by the... Text Text  Text Text.\vspace*{-12pt}

%\bibliographystyle{natbib}
%\bibliographystyle{achemnat}
%\bibliographystyle{plainnat}
%\bibliographystyle{abbrv}
%\bibliographystyle{bioinformatics}
%
%\bibliographystyle{plain}
%
%\bibliography{Document}


\begin{thebibliography}{}

\bibitem[Bofelli {\it et~al}., 2000]{Boffelli03}
Bofelli,F., Name2, Name3 (2003) Article title, {\it Journal Name}, {\bf 199}, 133-154.

\bibitem[Bag {\it et~al}., 2001]{Bag01}
Bag,M., Name2, Name3 (2001) Article title, {\it Journal Name}, {\bf 99}, 33-54.

\bibitem[Yoo \textit{et~al}., 2003]{Yoo03}
Yoo,M.S. \textit{et~al}. (2003) Oxidative stress regulated genes
in nigral dopaminergic neurnol cell: correlation with the known
pathology in Parkinson's disease. \textit{Brain Res. Mol. Brain
Res.}, \textbf{110}(Suppl. 1), 76--84.

\bibitem[Lehmann, 1986]{Leh86}
Lehmann,E.L. (1986) Chapter title. \textit{Book Title}. Vol.~1, 2nd edn. Springer-Verlag, New York.

\bibitem[Crenshaw and Jones, 2003]{Cre03}
Crenshaw, B.,III, and Jones, W.B.,Jr (2003) The future of clinical
cancer management: one tumor, one chip. \textit{Bioinformatics},
doi:10.1093/bioinformatics/btn000.

\bibitem[Auhtor \textit{et~al}. (2000)]{Aut00}
Auhtor,A.B. \textit{et~al}. (2000) Chapter title. In Smith, A.C.
(ed.), \textit{Book Title}, 2nd edn. Publisher, Location, Vol. 1, pp.
???--???.

\bibitem[Bardet, 1920]{Bar20}
Bardet, G. (1920) Sur un syndrome d'obesite infantile avec
polydactylie et retinite pigmentaire (contribution a l'etude des
formes cliniques de l'obesite hypophysaire). PhD Thesis, name of
institution, Paris, France.

\end{thebibliography}
\end{document}
